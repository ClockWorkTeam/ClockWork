%**VARIABILI PROGETTO DA MODIFICARE****************************
\def\DOCUMENTO       {Norme di Progetto}        %titolo del documento
\def\PROGETTO                   {\textbf{MyTalk}}       %titolo del progetto
\def\VERSIONE                        {0.1}                        %inserire qui la versione attuale (di base 0.1)
\def\CREAZIONE		{29/11/2012}              %data di creazione
\def\ULTIMA_MOD		{02/12/2012}              %data di ultima modifica
\def\USO_DOCUMENTO	{Interno}                      %Interno, Esterno

%Personale
%se ci sono pi\'u persone da indicare scrivere: {nome1, \\&nome2, \\&nome3 ecc..}
\def\REDATTORI		{Palmisano Maria Antonietta}
\def\VERIFICATORI	{}
\def\APPROVATORE	{}
\def\COMMITTENTI	{Gruppo {\textit{clockwork}} \\&Prof. Tullio Vardanega}

%Variabili documento
\def\TABELLE	{false} %abilita - disabilita l'indice delle tabelle
\def\FIGURE		{false} %abilita - disabilita l'indice delle figure


\documentclass[a4paper,11pt]{article}

%Non modificare
\def\CAPITOLATO       {\mbox{MyTalk}}
\def\SUBTITLE             {Sofware di comunicazione tra utenti senza requisiti di installazione}
\def\EMAIL                    {clockworkTeam7@gmail.com}
\def\NOME_GRUPPO               {\textit{clockwork}}
\def\PROPONENTE	   {}

%*****nuovi comandi************************
%fornisce il caption per riferirsi ad una particolare sezione
\newcommand{\numref}[1]{\textsf{\textsl{``\nameref{#1}'' (\ref{#1})}}}
%formattazione per il glossario
\newcommand{\gl}[1]{\underline{#1}}


%**IMPORTAZIONE PACKAGE**************************

\usepackage{ifthen}
\usepackage[italian]{babel}
\usepackage[utf8]{inputenc}
\usepackage[T1]{fontenc}
\usepackage{float}
\usepackage{chapterbib}
\usepackage{graphicx}
\usepackage[a4paper,top=3cm,bottom=3cm,left=3cm,right=3cm,bindingoffset=5mm]{geometry}
\usepackage[colorlinks=true, urlcolor=blue, citecolor=black, linkcolor=black]{hyperref}
\usepackage{booktabs}
\usepackage{fancyhdr}
\usepackage{totpages}
\usepackage{tabularx, array}
\usepackage{dcolumn}
\usepackage{epstopdf}
\usepackage{booktabs}   %pacchetto per le tabelle
\usepackage{fancyhdr}
\usepackage{longtable}
\usepackage{calc}
\usepackage{eurosym} % simbolo euro col comando \euro

\usepackage{datatool}
\usepackage[bottom]{footmisc}
\usepackage{listings} % used to report code


%**STILE PAGINA**********************************

\pagestyle{fancy}

%no indentazione paragrafo
\setlength{\parindent}{0pt}


%intestazione
\lhead{\includegraphics[keepaspectratio = true, width = 110px] {logo.png}}
\rhead{\Large{\bfseries{\PROGETTO}}}
\renewcommand{\headrulewidth}{0.4pt}  %Linea sotto l'intestazione

%pi� di pagina
\lfoot{{\DOCUMENTO} v {\VERSIONE}}
\rfoot{\thepage} %per le prime pagine: mostra solo il numero romano
\cfoot{}
\renewcommand{\footrulewidth}{0.4pt}   %Linea sopra il pi� di pagina



%**PRIMA PAGINA***************************
\begin{document}

%Prima pagina senza intestazione n� pi� di pagina									
\thispagestyle{empty}

%Centra il testo
\begin{center}
\vspace{1cm}

%Titolo del capitolato
\Huge{\textbf{\CAPITOLATO}} \\\Large{\textbf{\SUBTITLE}}

%Linea sotto il nome del capitolato
\rule{\textwidth}{0.2mm}	
 
%Spazio verticale
\vspace{2cm}

%Inserimento logo e e-mail
\includegraphics[keepaspectratio = true, width = 340px] {logo.png} \\ {\EMAIL}


%Spazio verticale
\vspace{0.7cm}

\Huge{\textbf{\DOCUMENTO}}

\vspace{0.7cm}

%Tabella con informazioni documento. 
\normalsize{
          \begin{tabular}{r|l}
                     \multicolumn{2}{c} {\textbf{Informazioni sul documento}} \\
	            \midrule
	            \textbf{Nome file}                                      & \DOCUMENTO \\
	            \textbf{Versione}                                       & v \VERSIONE \\
	            \textbf{Data creazione}                            & \CREAZIONE \\
	            \textbf{Data ultima modifica}                  & \ULTIMA_MOD \\
	            \textbf{Stato}                                              & Formale \\
	            \textbf{Uso}                                                & \USO_DOCUMENTO \\
	           \textbf{Redazione}                                   & \REDATTORI \\
      	          \textbf{Approvazione}                             & \APPROVATORE \\
	          \textbf{Verifica}                                         & \VERIFICATORI \\
	          \textbf{Distribuzione}                               & \COMMITTENTI \
     \end{tabular}
}

\vspace{0.7cm}

\textbf{Sommario} \\
Questo documento vuol definire le norme volte alla regolamentazione del gruppo \textit{clockwork}
 necessarie al processo di sviluppo del progetto \PROGETTO.
 
%Fine zona centrata
\end{center}

%Indica che � finita la pagina corrente ed inizia la prossima
\newpage
%si usa la numerazione romana per gli indici e la tabella delle modifiche
\pagenumbering{Roman}


%**TABELLA DELLE MODIFICHE***************************
%vari comandi per la struttura della tabella, NON MODIFICARE!
\begin{center}
\begin{tabular}{|c|p{5cm}|c|c|}
\hline
\textbf{Autore} & \textbf{Descrizione} & \textbf{Data} & \textbf{Versione} \\
\hline\hline
Palmisano Maria Antonietta & Inizio stesura documento & 29/11/2012 & 0.1 \\\hline
\hline
%*******************MODIFICA QUI*****************
%Scrivere qui la lista delle modifiche fatte al documento (dalla pi� recente alla pi� vecchia)
%tuttle le righe devono finire con \\\hline 
%Esempio:
% Cognome Nome & Modificato qualcosa & data & versione \\\hline
%*******************FINE MODIFICA****************
\end{tabular}
\end{center}
\newpage

%Inserisce il link all'indice
%\addcontentsline{toc}{section}{Indice}
\newpage
\tableofcontents
\newpage

%se � stata impostata a true la variabile per la lista delle tabelle, la mostra
\ifthenelse{\equal{\TABELLE}{true}} 
{\listoftables \newpage}{}

%se � stata impostata a true la variabile per la lista delle figure, la mostra
\ifthenelse{\equal{\FIGURE}{true}}
{\listoffigures \newpage}{}

%da qui comincia la numerazione normale
\pagenumbering{arabic}

%imposta il formato di visualizzazione
\rfoot{\thepage ~di~\pageref{TotPages}}



% Inizio delle sezioni di documentazione
\section{Introduzione}
\subsection{Scopo del documento}
\subsection{Ambiguit\`a}

\section{Comunicazioni}
\subsection{Comunicazioni interne}
\subsection{Comunicazioni esterne}

\section{Incontri}
\subsection{Incontri interni}
\subsubsection{Incontri esterni}

\section{Repository}
\subsection{Struttura repository}
\subsection{Utilizzo repository dei documenti}
\subsection{Utilizzo repository del codice}
\section{Documenti}
\subsection{Template}
\subsection{Norme tipografiche}
\subsubsection{Stile del testo}
\subsubsection{Punteggiatura}
\subsubsection{Composizione del testo}
\subsubsection{Formati ricorrenti}
\subsection{Componenti visive}
\subsubsection{Immagini}
\subsubsection{Tabelle}
\section{Versionamento}
\subsection{Formattazione documenti}
\subsection{Tipi di documenti}
\subsubsection{Verbali incontri}
\subsubsection{Documenti informali}
\subsubsection{Documenti formali}
\subsubsection{Procedura di formalizzazione dei documenti}
\section{Analisi dei Requisiti}
\section{Progettazione}
\section{Codifica}
\subsection{Intestazione file}
\subsection{Convenzioni di codifica}
\section{Glossario}
\subsection{Inserimento vocaboli}
\section{Protocollo per lo sviluppo dell'applicazione}
\section{Ambiente di progetto}
\subsection{Ambiente generale}
\subsubsection{Sistema operativo}
\subsection{Ambiente documentale}
\subsubsection{Scrittura documenti}
\subsubsection{Verifica ortografica}
\subsubsection{Pianificazione}
\subsubsection{Grafici UML}
\subsubsection{Documentazione semi-automatica}
\subsection{Ambiente sviluppo}
\subsubsection{Strumento di versionamento}
\subsubsection{Ambiente codifica}
\subsubsection{Framework}
\subsubsection{Design pattern}
\subsubsection{Front-end grafico}
\subsubsection{Piattaforma cloud}
\subsubsection{Ambiente mobile}
\subsection{Ambiente verifica e validazione}
\subsubsection{Requiseeds}
\subsubsection{Analisti statica}
\subsubsection{Test}
\subsubsection{Analisi dinamica}
\subsubsection{Validazione}

\end{document}























