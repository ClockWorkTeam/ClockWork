%**VARIABILI PROGETTO DA MODIFICARE****************************
\def\DOCUMENTO                    {Glossario}                 %titolo del documento
\def\PROGETTO                    {\textbf{MyTalk}}            %titolo del progetto
\def\VERSIONE                              {0.1}                        %inserire qui la versione attuale
\def\CREAZIONE		      {29/11/2012}               %data di creazione
\def\ULTIMA_MOD		       {01/12/2012}              %data di ultima modifica
\def\USO_DOCUMENTO	        {Esterno}                    %Interno, Esterno

%Personale
%se ci sono pi\'u persone da indicare scrivere: {nome1, \\&nome2, \\&nome3 ecc..}
\def\REDATTORI		{Palmisano Maria Antonietta}
\def\VERIFICATORI	{}
\def\APPROVATORE	{}
\def\COMMITTENTI	{Gruppo {\textit{clockwork}} \\&Prof. Tullio Vardanega}

\newcommand{\sortitem}[2]{%
  \DTLnewrow{list}%
  \DTLnewdbentry{list}{label}{#1}%
  \DTLnewdbentry{list}{description}{#2}%
}

\newenvironment{sortedlist}%
{%
  \DTLifdbexists{list}{\DTLcleardb{list}}{\DTLnewdb{list}}%
}%
{%
  \DTLsort{label}{list}%
  \begin{description}%
    \DTLforeach*{list}{\theLabel=label,\theDesc=description}{%
      \item[\textbf{\theLabel:}] \ \\ \theDesc
    }%
  \end{description}%
}

\documentclass[a4paper,11pt]{article}

%Non modificare
\def\CAPITOLATO       {\mbox{MyTalk}}
\def\SUBTITLE             {Sofware di comunicazione tra utenti senza requisiti di installazione}
\def\EMAIL                    {clockworkTeam7@gmail.com}
\def\NOME_GRUPPO               {\textit{clockwork}}
\def\PROPONENTE	   {}

%*****nuovi comandi************************
%fornisce il caption per riferirsi ad una particolare sezione
\newcommand{\numref}[1]{\textsf{\textsl{``\nameref{#1}'' (\ref{#1})}}}
%formattazione per il glossario
\newcommand{\gl}[1]{\underline{#1}}


%**IMPORTAZIONE PACKAGE**************************

\usepackage{ifthen}
\usepackage[italian]{babel}
\usepackage[utf8]{inputenc}
\usepackage[T1]{fontenc}
\usepackage{float}
\usepackage{chapterbib}
\usepackage{graphicx}
\usepackage[a4paper,top=3cm,bottom=3cm,left=3cm,right=3cm,bindingoffset=5mm]{geometry}
\usepackage[colorlinks=true, urlcolor=blue, citecolor=black, linkcolor=black]{hyperref}
\usepackage{booktabs}
\usepackage{fancyhdr}
\usepackage{totpages}
\usepackage{tabularx, array}
\usepackage{dcolumn}
\usepackage{epstopdf}
\usepackage{booktabs}
\usepackage{fancyhdr}
\usepackage{longtable}
\usepackage{calc}

\usepackage{datatool}
\usepackage[bottom]{footmisc}
\usepackage{listings} % used to report code


%**STILE PAGINA**********************************

\pagestyle{fancy}

%no indentazione paragrafo
\setlength{\parindent}{0pt}


%intestazione
\lhead{\includegraphics[keepaspectratio = true, width = 110px] {logo.png}}
\rhead{\Large{\bfseries{\PROGETTO}}}
\renewcommand{\headrulewidth}{0.4pt}  %Linea sotto l'intestazione

%pi� di pagina
\lfoot{{\DOCUMENTO} v {\VERSIONE}}
\rfoot{\thepage} %per le prime pagine: mostra solo il numero romano
\cfoot{}
\renewcommand{\footrulewidth}{0.4pt}   %Linea sopra il pi� di pagina



%**PRIMA PAGINA***************************
\begin{document}

%Prima pagina senza intestazione n� pi� di pagina									
\thispagestyle{empty}

%Centra il testo
\begin{center}
\vspace{1cm}

%Titolo del capitolato
\Huge{\textbf{\CAPITOLATO}} \\\Large{\textbf{\SUBTITLE}}

%Linea sotto il nome del capitolato
\rule{\textwidth}{0.2mm}	
 
%Spazio verticale
\vspace{2cm}

%Inserimento logo e e-mail
\includegraphics[keepaspectratio = true, width = 340px] {logo.png} \\ {\EMAIL}


%Spazio verticale
\vspace{0.7cm}

\Huge{\textbf{\DOCUMENTO}}

\vspace{0.7cm}

%Tabella con informazioni documento. 
\normalsize{
          \begin{tabular}{r|l}
                     \multicolumn{2}{c} {\textbf{Informazioni sul documento}} \\
	            \midrule
	            \textbf{Nome file}                                      & \DOCUMENTO \\
	            \textbf{Versione}                                       & v \VERSIONE \\
	            \textbf{Data creazione}                            & \CREAZIONE \\
	            \textbf{Data ultima modifica}                  & \ULTIMA_MOD \\
	            \textbf{Stato}                                              & Formale \\
	            \textbf{Uso}                                                & \USO_DOCUMENTO \\
	           \textbf{Redazione}                                   & \REDATTORI \\
      	          \textbf{Approvazione}                             & \APPROVATORE \\
	          \textbf{Verifica}                                         & \VERIFICATORI \\
	          \textbf{Distribuzione}                               & \COMMITTENTI \
     \end{tabular}
}

\vspace{0.7cm}

\textbf{Sommario} \\
Questo documento si prefigge di chiarire tutte le voci e le parole difficilmente comprensibili o dal significato ambiguo usate nella stesura dei documenti redatti dal gruppo {\textit{clockwork}}.


%Fine zona centrata
\end{center}

%Indica che � finita la pagina corrente ed inizia la prossima
\newpage
%si usa la numerazione romana per gli indici e la tabella delle modifiche
\pagenumbering{Roman}


%**TABELLA DELLE MODIFICHE***************************
%vari comandi per la struttura della tabella, NON MODIFICARE!
\begin{center}
\begin{tabular}{|c|p{5cm}|c|c|}
\hline
\textbf{Autore} & \textbf{Descrizione} & \textbf{Data} & \textbf{Versione} \\
\hline\hline
Palmisano Maria Antonietta & Inizio stesura documento & 29/11/2012 & 0.1 \\\hline
\hline
%*******************MODIFICA QUI*****************
%Scrivere qui la lista delle modifiche fatte al documento (dalla pi� recente alla pi� vecchia)
%tuttle le righe devono finire con \\\hline 
%Esempio:
% Cognome Nome & Modificato qualcosa & data & versione \\\hline
%*******************FINE MODIFICA****************
\end{tabular}
\end{center}
\newpage

%************************************************
%da qui comincia la numerazione normale
\pagenumbering{arabic}
%imposta il formato di visualizzazione
\rfoot{\thepage ~di~\pageref{TotPages}}

\newpage

%\section*{\Huge{A}}
%\begin{sortedlist}
  %\sortitem{parola}{definizione}
  %\sortitem{parola}{definizione}
%\end{sortedlist}
%\newpage



\section*{\Huge{A}}

\newpage 

\section*{\Huge{B}}

\newpage 



\end{document}

